\documentclass{article}
\usepackage[margin=1.0in]{geometry}
\usepackage[utf8]{inputenc}
\usepackage[colorlinks = true, urlcolor = blue]{hyperref}
\setlength{\parindent}{0pt}

\title{CS 6540 --- Human-Computer Interaction\\\textbf{Project 2: Think-Aloud Protocol}}
\author{ }
\date{\textbf{Due:} TBD}

\usepackage{natbib}
\usepackage{graphicx}

\begin{document}

\maketitle

\section{Summary}
\textbf{This is for educational purposes only. We do not have IRB approval to recruit participants for a real research study.}\\ 

This is an individual project.\\

The purpose of this assignment is to get hands-on experience with the think-aloud protocol. While it can sometimes be challenging to get participants to vocalize during the think-aloud, the think-aloud is a useful tool for uncovering (particularly first-time) interaction issues with systems.\\

The main idea here is that you will have multiple people perform a think-aloud session with the task given below. You will be turning in both the recorded sessions and a writeup based on the above points.\\
 
\textbf{Learning Objectives:}
\begin{enumerate}
    \item Gain a basic understanding of the think-aloud protocol.
    \item Gain some experience conducting think-aloud sessions, including:
    \begin{itemize}
        \item Recruiting
        \item Creating scripts for consent, confidentiality information, and any background questions
        \item Performing screen capture and audio recording
    \end{itemize}
    \item Reflect on the think-aloud protocol as a method, including:
    \begin{itemize}
        \item Strengths and weaknesses
        \item Sampling and representative users
        \item Intervening (or not) with the participant
    \end{itemize}
\end{enumerate}

\section{Resources}
We introduced the think-aloud protocol in class and watched a video of an example think-aloud session. As mentioned in the slides, \textit{Universal Methods of Design} (\href{https://utah-primoprod.hosted.exlibrisgroup.com/primo-explore/fulldisplay?docid=UUU_ALMA21274580350002001&context=L&vid=UTAH}{available online via the University of Utah library}) has an entry on the Think-Aloud Protocol. \\

You might also find Boren and Ramey's \href{https://ieeexplore.ieee.org/document/867942/}{``Thinking aloud: reconciling theory and practice''} useful both in terms of raising issues that you might run into and providing pointers to further resources. Those interested in the relationships between methods and the theories that support them (see \href{https://uxdesign.cc/method-vs-methodology-whats-the-difference-9cc755c2e69d}{the difference between a method and a methodology}) might also appreciate the article.\\

For the purposes of this assignment, you do not have to exactly follow Boren and Ramey's recommendations or Ericsson and Simon's methodology. We are more interested in what works for you and what you learn about the process.\\

There is not necessarily one ``right'' way to do the think-aloud; in general for methods, whether or not you're ``doing it right'' depends on what you're trying to get out of it and what you're trying to claim. For example: are you trying to say that your results indicate what users do in a completely unbiased and uninfluenced interaction? Are you trying to say that your method is backed by published theory X on how people think or interact? If you want to claim those things, then yes, there are ways that are or are not correct to perform the method. Or are you trying to claim that your results identify all possible interface problems? In which case, \textit{conducting some number of think-aloud sessions will never be the right method}.

\section{Overview of Your Tasks}
\begin{enumerate}
    \item Determine how you will get participants to help you out with the project.
    
    \item Establish exact scripts for how you will introduce and explain the task to the participant. Those scripts should include the items mentioned below (data retention and privacy, consent regarding recording).
    
    \item If you believe that they would be helpful, identify reasonable questions to ask the participant before they begin. They may not be too invasive in terms of privacy, but could establish ``useful'' information for you regarding the participants' background as it pertains to the task. These questions must have exact wordings: do not say ``I asked them about X.'' You must have a script for exact questions. \textit{To clarify: this project is not meant to be an interview. It is meant to be a think-aloud protocol. Keep any questions brief.}
    
    \item Conduct multiple (concurrent) recorded think-aloud sessions. \textit{To clarify: we mean concurrent in the sense that the participant speaks while conducting the task, not that you run multiple think-aloud sessions at the same time. }In the course of the sessions you will:
    \begin{enumerate}
        \item Begin recording.
        \item Inform the participant that you are recording (screen and audio) and how you will handle the recordings.  
        \item Obtain consent.
        \item Ask the participant any scripted questions and use your script(s) to introduce the task, in whichever order you prefer.
        \item Have the participant perform the task.
        \item Conduct any concluding business.
        \item Stop recording.
    \end{enumerate}

    \item Complete the writeup (which includes analyzing the think-aloud sessions). 
    
    \item Turn in the recorded sessions and the writeup.

\end{enumerate}

\section{Other Rules and Guidelines}
\begin{itemize}
    \item You must conduct think-aloud sessions with \textbf{at least 4 different participants}. The total combined duration of your recorded sessions must be \textbf{at least 1 hour}. So, you might have 4 participants whose sessions last a total of 2 hours, or you might need 5 participants in order to have a combined total of 1 hour. \textbf{You must have at least 2 participants who are not CS 6540 classmates. Classmates from other classes are fine and count as non-CS-6540.} \textit{If you are finding that it's looking like it will take too many (6+?) participants to fulfill the 60-minute requirement, please reach out to us. You might not be getting the participants to verbalize enough or you might want to target first-time users unfamiliar with the interface.}
    
    \item You must record sessions via a screen capture with an audio track of what the person is saying. You are responsible for determining how you want to do so, but there are multiple free or demo options that work well. As an idea, a 30-minute screen capture with audio that I took on my laptop with a free tool is approximately 1.8GB. 
    
    \item Determine how you will handle the recordings with respect to privacy and data retention. You must inform the participant that you are recording (screen and audio) and receive verbal consent from the participant that they will have the session recorded. Explain to the participant how you will handle the recordings. Any copies that we make of the recordings will be kept on password-protected systems accessible only by the course staff. The recordings will be used only for educational purposes and will be destroyed after the semester is over. \textbf{Do not record video of the participant.}

    \item Participants should be anonymous. Don't tell us who they are.

    \item You don't need to transcribe your sessions, but you may choose to do so if it helps you analyze them. Be warned that transcription can be time consuming.

    \item Use your own computer, not the participant's --- we want to avoid any potential privacy issues from screen capture, and they won't necessarily be set up for screen capture anyway.
    
\end{itemize}

\section{Their Task}
Have participants go to \href{https://www.kobo.com/}{https://www.kobo.com/} (Rakuten Kobo eBooks) and choose a free eBook that they might want to read. They should download the eBook (they may need to create and account and download software) and read the first few pages. It's up to you whether they read silently or out loud --- we just want to force them to open the (downloaded) book and flip pages.

\section{To Turn In}
You will be turning in:
\begin{enumerate}
    \item \textbf{[40 points]} All of your recorded sessions (the screen capture with the recorded audio of the participant speaking). \textbf{Do not use Canvas to turn in these files.} We suggest that you use \href{https://gcloud.utah.edu/}{https://gcloud.utah.edu/}, which should offer you unlimited storage. Keep the recorded sessions private (not publicly viewable), but please share them with both \href{mailto:ahmad.alsaleem@utah.edu}{ahmad.alsaleem@utah.edu} and \href{mailto:tdenning@cs.utah.edu}{tdenning@cs.utah.edu}.
    
    \item The writeup of this assignment, as a PDF, to Canvas. As indicated below, all the writeup questions total 77 points.
    
\end{enumerate}

\section{The Writeup}
As mentioned above, the writeup addresses two different topics: the UX from the task (30 points) and the think-aloud method itself (47 points). In general we aren't as much looking for the ``right'' answer as for thoughtful reflection. There may be some redundancy in your answers, but in general attempt to use the question answers to convey any interesting insights that you encountered during the assignment.

\subsection{Task UX}
\begin{enumerate}
    \item \textbf{[10 points]} Describe every problem that your participants ran into. ``Problem'' doesn't mean that they were unable to complete the task, but can be any aspect of the interface or interaction with which they indicated confusion or dissatisfaction, by either speech or action. If multiple participants ran into the same problems, don't document them twice --- just indicate how many participants had that problem. As much as possible, document with screenshots.
    
    \item \textbf{[10 points]} Document aspects of the experience or interface that were particularly smooth or received praise from participants. If multiple participants had positive experiences with the same aspects, don't document them twice --- just indicate how many participants had that experience. As much as possible, document with screenshots.
    
    \item \textbf{[10 points]} Describe \textbf{2 (different)} concrete suggestions for improvement for the interface/interaction. If applicable, supplement with sketches or mockups (e.g., insert photos of drawings, or take a screenshot of something you put together in an image editor). We will not grade you on the ``slickness'' of the supporting image, but instead on the suggestion for improvement (and how well the supporting image conveys the idea behind that suggestion). Ideally (but not necessarily) these suggestions should address issues encountered by your participants.

    \item \textbf{[0 points]} Were there any things that your participants had problems with that you weren't expecting? What were they?

    \item \textbf{[0 points]} Were there any things that you expected your participants to have problems with that they didn't have problems with? What were they?

\end{enumerate}

\subsection{Think-Aloud Method Reflection}
\begin{enumerate}
    \item \textbf{[10 points]} Give all the script(s), exactly, that you used with participants. They must include explaining the recording (including privacy and retention), obtaining consent, and explaining the task. If you asked participants any background questions give the exact scripts for those questions.
    
    \item \textbf{[2 points]} How did you get participants to participate? Were there any difficulties?
    
    \item \textbf{[2 points]} Lay out, in as much detail as possible, who you think are representative users of the system. Are the people who did think-alouds for you representative users? Why or why not?
    
    \item \textbf{[2 points]} If you wanted to recruit a large number of (as representative as possible) further users for this think-aloud task, how would you do so? Why?
    
    \item \textbf{[2 points]} \textit{If you asked the participants questions,} why did you ask the questions you asked? Were there questions that you wanted to ask but didn't? What were they? Why didn't you ask them? \textit{If you didn't ask the participants questions,} why not?
    
    \item \textbf{[4 points]} Did you change anything about how you conducted the think-aloud between sessions? If so, what? Why or why not?
    
    \item \textbf{[2 points]} Did you end up answering any questions from the participants? \textit{If so}, what, and why? In retrospect, should you have handled the situation a different way? \textit{If not}, why not?
    
    \item \textbf{[2 points]} Did you end up saying anything (unprompted) to any participant during the task? \textit{If so}, what, and why? In retrospect, should you have handled the situation differently? \textit{If not}, why not?

    \item \textbf{[4 points]} Overall, \textbf{in terms of the think-aloud method}, what worked well? What didn't work well?
    
    \item \textbf{[3 points]} When might you decide to use a think-aloud in the future, and why? When would you not use a think-aloud task, and why?

    \item \textbf{[4 points]} What are the strengths of the think-aloud method? What are its weaknesses?

    \item \textbf{[0 points]} Did you look up further resources to help you perform your think-aloud project? If so, what were they? Were they helpful? Why or why not?

    \item \textbf{[10 points]} Imagine that you are giving advice to someone else (a researcher, new UX professional) who is learning how to do a think-aloud with participants. In approximately 500 words, give them advice and suggest some best practices for a think-aloud. Make as much of the advice as possible the things that ``aren't obvious'' or ``are surprising.'' You can assume that they have read a basic description of the method, so paraphrasing ``you get the participant to think aloud while completing a task," etc., is \textit{not} helpful.

\end{enumerate}

\section{Other Notes}
\begin{itemize}
    \item \href{https://en.wikipedia.org/wiki/Convenience_sampling}{Convenience sampling} (which can get you in trouble in many circumstances and is almost a dirty phrase) is 100\% okay for this project. We care more about your getting hands-on experience and reflecting on it than on getting data from ``representative users.'' It's fine to have a person who isn't and absolutely would never be a user of the system --- just note that they aren't representative and why you think they aren't representative (for this task, admittedly, the audience is pretty broad).

    \begin{itemize}
        \item As a first resource, I'd say go straight to your CS 6540 classmates for an exchange --- they do yours and you'll do theirs. Potentially also useful to see how others run their sessions. Your classmates will probably(?) also tend to be better about thinking aloud since they know that they will have / are having a hard time getting people to talk in their own sessions.
    
        \item But for the remaining two! Totally okay: friends, family, roommates, classmates from other courses.
        
        \item \textit{If you've exhausted the above}, feel free to get in contact with us. There are ways to put out calls/requests to grads and undergrads that you don't know for their assistance. While we don't plan to give you funds to compensate participants, I am open to helping you get bribe donuts, etc. This is the most complicated logistically, so I wouldn't start with this option, but the purpose of this course also isn't to see how many friends you have or make your friends mad at you.
        
    \end{itemize}
\item I'd suggest getting at least tentative commitments from people earlier rather than later so that you aren't scrambling last minute to get people.

\item A ``representative user'' of your system is a typical (or atypical) expected (or demonstrated) user type for your system. What are their characteristics? How would you describe them? This can be speculative or data-driven.

\item In some situations you would \textit{want} to record video of the participant: e.g., if you plan to perform manual or automated analysis of their facial expressions with respect to task steps. However! We aren't doing that in this project and therefore video capture would be an unnecessary invasion of privacy and logistically would increase the size of capture files.
\end{itemize}
\end{document}
