\documentclass{article}
\usepackage[margin=1.0in]{geometry}
\usepackage[utf8]{inputenc}
\usepackage[colorlinks = true, urlcolor = blue]{hyperref}
\usepackage{csquotes}
\usepackage{tikzsymbols}
\setlength{\parindent}{0pt}

\title{CS 6540 --- Human-Computer Interaction\\\textbf{Project 4: Literature Review}}
\author{ }
\date{\textbf{Due:} TBD}

\usepackage{natbib}
\usepackage{graphicx}

\begin{document}

\maketitle

\section{Summary}
\textbf{Note:} As previously mentioned, this is a class that is partially focused on giving you \textit{hands-on experience} with \textit{skills that are useful} for conducting HCI research. As a result, this handout sometimes uses the term `research' as an adjective for, e.g., your focus topic for your group projects. We wish to emphasize that despite the occasional usage of that term, the purpose of these projects is entirely educational; we are not actually conducting research (i.e., a systematic investigation designed to develop or contribute to generalizable knowledge) and any findings from these group projects are not publishable. Review from a human subjects board is necessary in order to perform human subjects research (as opposed to classroom assignments, which are meant to increase your personal expertise and knowledge).\\

This is the first of four group projects.\\

You should form a group of 2--3 people for this project and keep that same group for all group projects. Additionally, you should plan to use the same research topic/research questions for all the group projects. \textit{Note:} This does not mean that your research topic/research questions cannot evolve or become more specific over the course of the projects; however, you should not change topics entirely between projects.\\

The purpose of this assignment is to get hands-on experience with conducting a literature review of related work.\\

\textbf{Learning Objectives:}
\begin{enumerate}
    \item Be able to articulate what a literature review is and what it is for.
    \item Be able to locate relevant related work.
    \item Be able to articulate, for a given publication, why it is relevant to the research project and how it is different from the proposed work.
    \item Be able to reflect on the process of conducting a literature review.
\end{enumerate}\\


\section{Background and Resources}
Chapter 3 (specifically, pages 19--25) of \textit{Writing for Computer Science} by Justin Zobel (\href{https://ebookcentral-proquest-com.ezproxy.lib.utah.edu/lib/utah/detail.action?docID=1974126}{available online via the University of Utah library}) is a nice starting resource on conducting a literature review. The following is from that chapter:

\begin{displayquote}
\textit{Your reading achieves several aims. It establishes that your work is indeed novel or innovative; it helps you to understand current theory, discoveries, and debates; it can identify new lines of questioning or investigation; and it should provide alternative perspectives on your work. This reading will ultimately be summarized in the background sections and the discussions of related work in your write-ups.}
\end{displayquote}

Note that a literature review in this sense---finding and reporting on prior published research related to the project---is \textit{not} the same as reviewing a paper for a conference or a journal.\\

The reading supplies a number of tips for finding papers, including:
\begin{itemize}
    \item Searching using relevant terms on \href{https://dl.acm.org/}{the ACM digital library}, \href{https://ieeexplore.ieee.org/Xplore/home.jsp}{the IEEE Xplore digital library}, and \href{https://scholar.google.com/}{Google Scholar}. Note that you may start with the closest terms that you can think of for the research, then may discover as you look through resulting papers that there is specific lingo that would be more effective in future searches.
    \item Using a promising paper as a springboard by:
    \begin{itemize}
        \item Looking through all papers cited in that paper's reference section.
        \item Looking through all papers that cite \textit{that} paper (the above digital libraries show that information on a paper's listing).
        \item Looking through other publications from the paper's authors (most usually the first or the last author).
    \end{itemize}
    \item Browsing through the technical programs from the last year or two of top HCI conferences (e.g., CHI, UIST, CSCW, Ubicomp, DIS).
\end{itemize}

The Zobel chapter gives other good advice, including:
\begin{displayquote}
\textit{A thorough search of the literature can easily lead to discovery of hundreds of potentially relevant papers...there is rarely a need to understand every line...it is important to become an effective reader, by giving each paper neither more nor less time that it deserves. The first time you read a paper, skim through it to identify the extent to which it is relevant—only read it thoroughly if there is likely to be value in doing so...Expect to have a range of modes of reading: browsing to find papers and get an overview of activity and to understand the main outcomes in a body of work; background reading of texts and popular science magazines; and thorough, focused reading of key or complex papers that stretch your abilities or the limits of your understanding. But don’t allow reading to develop into a form of procrastination---it needs to be part of a productive cycle of work, not a dominant use of time.}
\end{displayquote}

\section{Your Assignment Tasks}
\begin{enumerate}
    \item Start the writeup by articulating your topic or research questions.
    \item Conduct as a group a literature review for your project.
    \item Complete and turn in the writeup.
\end{enumerate}

\section{To Turn In}
You will be turning in:
\begin{enumerate}
    \item The writeup of this assignment, as a PDF, to Canvas. Include all group member names on the writeup. Turn in only one writeup per group. As indicated below, all the writeup questions total 196 points.
    
\end{enumerate}

\section{The Writeup}
The writeup addresses two different topics: the related works that your group found (160 points) and a reflection on the literature search/review process (36 points). In general we aren't as much looking for the ``right'' answer as for thoughtful reflection. There may be some redundancy in your answers, but in general attempt to use the question answers to convey any interesting insights that you encountered during the assignment.

\subsection{Related Works}
In general we will be grading this section of the writeup based on how relevant the related works are to your topic/questions, how well you articulate their relevance/differences, and the venues from which your references originate. Publications should be from reasonable peer-reviewed academic research venues. As a rough rule of thumb, at least 3/4 of the publications should be from either CHI, UIST, CSCW, Ubicomp, or DIS. Feel free to reach out to us with any questions about exceptions---for example, perhaps the work has to do with accessibility and therefore most of your citations are from ACM ASSETS.
\begin{enumerate}
    \item Give information on the \textbf{15 academic research publications} that you found that are \textit{most relevant} to your work. For each publication:
    \begin{itemize}
        \item \textbf{[30 points]} Give the full citation for the reference. (ACM DL, IEEE Xplore, etc. have easy citation export links.) In general a full citation means: All author names. The paper or article title. The full title of the conference or journal in which it appears (including volume and issue, if appropriate). The year published.
        \item \textbf{[30 points]} In at least 100 words, give a summary of the publication.
        \item \textbf{[30 points]} Why is the publication relevant?
        \item \textbf{[30 points]} How is the publication different from your (proposed) work?
    \end{itemize}
    \item Give information on \textbf{5 other publications} that you found that are \textit{not as relevant} as the above publications. For each publication:
    \begin{itemize}
        \item \textbf{[10 points]} Give the full citation for the reference.
        \item \textbf{[10 points]} In at least 100 words, give a summary of the publication.
        \item \textbf{[10 points]} Why is the publication relevant? Alternatively, how and why did you run across it in your search?
        \item \textbf{[10 points]} Why is the publication less relevant than the above publications?
        \end{itemize}
    \item \textbf{[0 points]} Are there any other particularly relevant items that don't come from the academic literature? For example, motivating news articles, industry products, or non-peer-reviewed whitepapers/technical reports.
\end{enumerate}

\subsection{Literature Review Reflection}
\begin{enumerate}
    \item \textbf{[6 points]} \textbf{At the \textit{start} of the project, before you begin your literature search:} In at least 100 words, describe your research topic/research questions. This may be as specific as exact research questions that you are seeking to answer in the following projects or a more general research area (e.g., end-user mental models for computer security).
    \item \textbf{[6 points]} \textbf{At the \textit{end} of the project, after you have finished (for the time being) your literature search:} In at least 100 words, describe your research topic/research questions. 
    \item \textbf{[4 points]} Did you change your research topic/questions? If so, why? If not, why not?
    \item \textbf{[4 points]} Which search terms were most effective in finding work related to your research?
    \item \textbf{[4 points]} What other terms did you use in your search?
    \item \textbf{[4 points]} If you were to spend more time working on your literature review, what would be your next steps? If you think that spending more time wouldn't yield useful results, why?
    \item \textbf{[8 points]} Imagine that you are giving advice to a researcher who is learning how to do a literature review. In approximately 250 words, give them advice and suggest some best practices. Make as much of the advice as possible the things that ``aren't obvious'' or ``are surprising.'' You can assume that they have read a basic description of how to conduct a literature review, so paraphrasing the description is not helpful.
\end{enumerate}

\section{Other Notes}
\begin{itemize}
    \item ``There are no related works'' \textbf{is never the right answer}. It may be true that there aren't \textit{closely} related works. If that's the case, you could interpret that as either good news or bad news!
    \begin{itemize}
        \item The Good News interpretation: The idea is either very new (congratulations on being on the cutting edge of research!) or novel (congratulations on being creative!).
        \item The Bad News interpretation: Maybe you just didn't look hard enough or your idea is unimportant. \Laughey[1.4]
    \end{itemize}
    In either case, a literature review would articulate the most relevant related works and why they are different from your work. \textbf{As a metaphor: }there may well be a hole in published research which your research would fill, but then you need to let others understand what the shape of that hole is. That means that you need to indicate what the edges look like.
    
    \item When conducting research and writing papers, the types of related works you focus on may vary depending on circumstances. For instance, in some cases it might be more or less appropriate to focus on:
    \begin{itemize}
        \item Other work in the same domain area.
        \item Work in other domain areas that looks at similar problems.
        \item Work that uses similar methodologies.
    \end{itemize}
    
    \item If you're finding that you don't have anything very thoughtful or detailed to say regarding a publication's relevance/difference from your project---or perhaps if all of the relevant/different descriptions are looking very similar---it might be a sign that your research topic/research questions are too broad.
    
    %\item \textit{How much is enough?} It can be particularly challenging to know when to stop if you aren't finding anything very closely related and you don't know the area well. 
\end{itemize}

\end{document}
