\documentclass{article}
\usepackage[margin=1.0in]{geometry}
\usepackage[utf8]{inputenc}
\usepackage[colorlinks = true, urlcolor = blue]{hyperref}
\usepackage{csquotes}
\usepackage{tikzsymbols}
\setlength{\parindent}{0pt}

\title{CS 6540 --- Human-Computer Interaction\\\textbf{Project 7: Final Group Project Report}}
\author{ }
\date{\textbf{Due:} TBD}

\usepackage{natbib}
\usepackage{graphicx}

\begin{document}

\maketitle

\section{Summary}
\textbf{Note:} As previously mentioned, this is a class that is partially focused on giving you \textit{hands-on experience} with \textit{skills that are useful} for conducting HCI research. As a result, this handout sometimes uses the term `research' as an adjective for, e.g., your focus topic for your group projects. We wish to emphasize that despite the occasional usage of that term, the purpose of these projects is entirely educational; we are not actually conducting research (i.e., a systematic investigation designed to develop or contribute to generalizable knowledge) and any findings from these group projects are not publishable. Review from a human subjects board is necessary in order to perform human subjects research (as opposed to classroom assignments, which are meant to increase your personal expertise and knowledge).\\

This is the fourth and final group project.\\

The purpose of this assignment is to get hands-on experience with writing about research.\\

\textbf{Learning Objectives:}
\begin{enumerate}
    \item Be able to describe your research questions, procedure, and results in a style appropriate for academic publication.
\end{enumerate}

\section{Resources}
In general, the readings that we have had throughout the semester should help you understand how to write a research paper. \textit{Writing for Computer Science} is helpful. You might also enjoy resources like the following:

\begin{itemize}
\item Philip Guo likes:
    \begin{itemize}
        \item \href{https://faculty.washington.edu/ajko/advice#goodpaper}{How do I write a good research paper}
        \item \href{http://faculty.washington.edu/wobbrock/pubs/Wobbrock-2015.pdf}{Catchy Titles Are Good: But Avoid Being Cute}
    \end{itemize}
\item From Matt Welsh:
\begin{itemize}
    \item \href{http://matt-welsh.blogspot.com/2016/04/why-i-gave-your-paper-strong-reject.html}{Why I gave your paper a Strong Reject}
    \item \href{http://matt-welsh.blogspot.com/2016/04/why-i-gave-your-paper-strong-accept.html}{Why I gave your paper a Strong Accept}
\end{itemize}
\end{itemize}

\section{Your Assignment Tasks}
\begin{enumerate}
    \item Following the structure of a CHI submission (\href{http://chi2019.acm.org/authors/chi-proceedings-format/}{http://chi2019.acm.org/authors/chi-proceedings-format/}---do not use the Extended Abstracts format), write a final project report for your group project as if it were a research paper.
    
\end{enumerate}

\section{Other Details and Requirements}

\begin{itemize}
    \item Your group submission should be approximately 6 pages for a two-person team and approximately 9 pages for a three-person team (excluding references).
    \item Feel free to include things that are appropriate for a course project discussion that you would not include in an actual research submission. For example: feel free to discuss what you would have done differently if you had not been under various constraints.
\end{itemize}

\section{To Turn In}
You will be turning in:
\begin{enumerate}
    \item A PDF of your final report to Canvas. Include all group member names on the writeup. Turn in only one writeup per group.  
\end{enumerate}

\end{document}
