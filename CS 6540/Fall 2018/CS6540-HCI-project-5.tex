\documentclass{article}
\usepackage[margin=1.0in]{geometry}
\usepackage[utf8]{inputenc}
\usepackage[colorlinks = true, urlcolor = blue]{hyperref}
\usepackage{csquotes}
\usepackage{tikzsymbols}
\setlength{\parindent}{0pt}

\title{CS 6540 --- Human-Computer Interaction\\\textbf{Project 5: Conducting Interviews}}
\author{ }
\date{\textbf{Due:} TBD}

\usepackage{natbib}
\usepackage{graphicx}

\begin{document}

\maketitle

\section{Summary}
\textbf{Note:} As previously mentioned, this is a class that is partially focused on giving you \textit{hands-on experience} with \textit{skills that are useful} for conducting HCI research. As a result, this handout sometimes uses the term `research' as an adjective for, e.g., your focus topic for your group projects. We wish to emphasize that despite the occasional usage of that term, the purpose of these projects is entirely educational; we are not actually conducting research (i.e., a systematic investigation designed to develop or contribute to generalizable knowledge) and any findings from these group projects are not publishable. Review from a human subjects board is necessary in order to perform human subjects research (as opposed to classroom assignments, which are meant to increase your personal expertise and knowledge).\\

This is the second of four group projects.\\

You should use the same group from the previous project for this project. Additionally, you should use (roughly) the same research topic/research question as in the previous project. \textit{Note:} This does not mean that your research topic/research questions cannot evolve or become more specific over the course of the projects; however, you should not change topics entirely between projects.\\

The purpose of this assignment is to get hands-on experience with writing interview questions, conducting interviews, and transcribing interviews.\\

\textbf{Learning Objectives:}
\begin{enumerate}
    \item Be able to write neutral, useful interview questions based on a research topic and research questions.
    \item Be able to conduct semi-structured interviews.
    \item Be able to transcribe interviews.
    \item Be able to reflect on the process of drafting, conducting, and transcribing an interview.
\end{enumerate}\\

\section{Your Assignment Tasks}
\begin{enumerate}
    \item Draft interview questions.
    \item Establish recruitment criteria and strategies and recruit participants.
    \item Conduct interviews.
    \item Partially transcribe interviews.
    \item Complete and turn in the writeup. Also turn in the interview recordings and transcriptions.
\end{enumerate}

\section{Other Details and Requirements}

\begin{itemize}
    \item \textbf{Each member of your group} must conduct \textbf{at least 2 interviews}. 3--4 interviews per group member is preferable. Additionally, we require that each group member be present at \textbf{at least 2 other interviews} to observe and take notes. Please have a maximum of 2 members present at interviews.
    
    \item Overall, your group must have \textbf{at least 1 cumulative hour of interview time per group member}.

    \item As with Project 2 (Think-Aloud Protocol), you will be recording participant sessions. For this project we are asking for audio recordings only (not video or screen capture). As with Project 2, think about any privacy or data retention issues and prepare any scripts, if necessary, for explaining the situation to the participants and obtaining consent.
    
    \item \textbf{Each group member} must transcribe \textbf{at least 15 minutes} of interview audio. Everyone should transcribe a different 15-minute time window.
    
    \item You should aim to have interviews take between 30 and 60 minutes. There is no set number of questions you need to ask---papers that we have read include their interview questions and how long the interviews took, which can help you calibrate.
    
    \item While you shouldn't wildly change your interview questions between interviews, you should feel free to change, add, or drop questions as you proceed. As much as possible, this should only be done in collaboration with your group members.
    
    \item Participants can remain anonymous---we don't need to know who they are. 
    
    \item Participants must be 18+.
    
    \item In general, you can ask any questions that:
    \begin{itemize}
        \item Are ethical.
        \item Have a point (What will you learn from them? Is it interesting? Would other researchers care?).
        \begin{itemize}
            \item More particularly, \textbf{are interview questions the best (or a reasonable) way to get at the information that you want to know?} If the answer is, ``actually, the research questions would best be answered by analyzing user logs and asking people is a poor substitute,'' then you should \textit{come up with (related) research questions that are more suited to being answered by interviews}.
            \item Imagine that you're writing a paper (like those we've read) that details the research questions, interview questions, and (eventually) results. \textbf{Would you find the paper convincing?}
        \end{itemize}
        \item Have something to do with technology.
        \item Have something to do with each other and your literature search.
    \end{itemize}
    Keep in mind that the results from your literature survey, this project, and the upcoming questionnaire project will be combined into one coherent written report for the final group project.
\end{itemize}

\section{To Turn In}
You will be turning in:
\begin{enumerate}
    \item \textbf{[60 points]} All of your recorded sessions. Each audio file should have an accompanying file with the exact interview materials (script, questions) that were used to conduct that specific interview. \textbf{Do not use Canvas to turn in these files.} We suggest that you use \href{https://gcloud.utah.edu/}{https://gcloud.utah.edu/}, which should offer you unlimited storage. Keep the recorded sessions private (not publicly viewable), but please share them with both \href{mailto:u0934311@gcloud.utah.edu}{u0934311@gcloud.utah.edu} and \href{mailto:tdenning@cs.utah.edu}{tdenning@cs.utah.edu}.
    \item \textbf{[15 points]} The partial interview transcripts. \textbf{Each member} should turn in their own transcript. They should indicate which file is being transcribed and what time frame in the file the transcription covers. They should also indicate which interviews they conducted and which interviews they attended as an observer/note-taker. 
    \item The writeup of this assignment, as a PDF, to Canvas. Include all group member names on the writeup. Turn in only one writeup per group. As indicated below, all the writeup questions total 85 points.
    
\end{enumerate}

\section{The Writeup}
The writeup is a reflection on drafting, conducting, and transcribing semi-structured interviews. In general we aren't as much looking for the ``right'' answer as for thoughtful reflection. There may be some redundancy in your answers, but in general attempt to use the question answers to convey any interesting insights that you encountered during the assignment.

\begin{enumerate}
    \item \textbf{[25 points]} Give the final version of any interview questions and scripts that you used to conduct the interviews. Also include recruitment materials (e.g., email text). If the materials changed substantially over the course of the project, then give the original drafts as well.
    
    \item \textbf{[8 points]} Give the research questions that were you trying to address via your interview. Did the answers from the interview questions help answer the research questions? Why or why not?
    
    \item \textbf{[6 points]} How did the interview materials change (or not change) over time? Why?
    
    \item \textbf{[6 points]} If you had to categorize your interview questions into groups that represent interview sub-topics, what would they be?
    
    \item \textbf{[8 points]} How did you go about trying to write ``good'' interview questions? In addition to indicating any general properties you tried to achieve, give some examples of exact question wordings and what motivated them.

    
    \item \textbf{[4 points]} What were inclusion criteria for participation in your study? What were exclusion criteria?
    
    \item \textbf{[4 points]} How did you recruit for your interviews? What worked well? What didn't work well?
    
    \item \textbf{[2 points]} If you were conducting these interviews for research (instead of coursework), would you recruit differently? Why or why not?
    
    \item \textbf{[2 points]} If you were conducting these interviews for research, are there any other things that you would have done differently? Why or why not?
    
    \item \textbf{[2 points]} Did group members feel that they learned anything from observing each other? Why or why not?
    
    \item \textbf{[2 points]} Did group members feel that they learned anything from transcribing interview audio? Why or why not?

    \item \textbf{[0 points]} Did you use other resources to learn about drafting, conducting, or transcribing semi-structured interviews? If so, what were they, and were they helpful?
    
    \item \textbf{[4 points]} Were there other research topics or research questions that you (as a group)  would rather have pursued if you weren't required to conduct interviews?
    
    \item \textbf{[2 points]} Were there any ways that you felt the format and requirements of this project limited questions that you would want to ask, if you were pursuing this project for research purposes?
    
    \item \textbf{[5 points]} Why are your research questions/interview subtopics/interview questions \textbf{well-suited to an interview method}, specifically? Why are they not better addressed by an eyetracking study/questionnaire (N$=$500)/log analysis/etc.?
    
    \item \textbf{[5 points]} Why are your research questions/interview subtopics important? Who cares, and why?
\end{enumerate}

\section*{Notes}
\begin{itemize}
    \item Distance collaborations are alright! Specifically:
    \begin{itemize}
        \item You can run interviews remotely (phone or video chat) or recruit remote participants.
        \item If it is difficult to coordinate attending group members' interviews, you can instead listen to the audio recordings of the interviews. (It is, however, better to attend in person---you'll both be able to see body language and observe interactions that take place before recording starts.)
    \end{itemize}
    \item There are many blog posts on transcribing, e.g.: \href{https://qualpage.com/2017/01/19/tips-on-transcribing-qualitative-interviews-and-naturally-occurring-interaction/}{https://qualpage.com/2017/01/19/tips-on-transcribing-qualitative-interviews-and-naturally-occurring-interaction/}
    \item For \textbf{every interview question}, you should ask yourself and be able to answer:
    \begin{itemize}
        \item Why are we asking this question?
        \item What are we hoping to learn from any answers?
        \item (If your answers to the above two points don't closely match the interview questions you're asking, can you reasonably ask the question that you actually \textit{mean}?)
        \item If we wrote up something about the question answers in a paper, what would it look like? What would be able to claim? What would we \textit{not} be able to claim?
        \item Can you think of ways to misunderstand or be confused about the question (definitions of terms, what ``this'' refers to, taking out of context, etc.)? (Interview pilots, of course, are very useful for this reason!)
    \end{itemize}
    
\end{itemize}

\end{document}
